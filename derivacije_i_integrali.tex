\documentclass[12pt,oneside,a4paper]{report}
% \linespread{1.3}
\title{derivacije i integrali}
% \renewcommand{\familydefault}{cmss}
\renewcommand{\arraystretch}{1.9}
\usepackage[croatian]{babel}
% \usepackage[dvips]{color}
% \usepackage[dvips]{rotating}
% \usepackage{epsfig}
\usepackage{floatflt}
\addtolength{\hoffset}{-2cm}
\addtolength{\voffset}{-2.5cm}
\addtolength{\textwidth}{3cm}
\addtolength{\textheight}{4cm}
\pagestyle{empty}

\begin{document}
\noindent

\begin{center}
\fbox{\fbox{{\Large \textbf{Derivacije}}}}
\end{center}

\vspace{5mm}
\noindent
\underline{\textbf{Osnovna pravila deriviranja:}}\\
\begin{itemize}
\item derivacija konstante: \fbox{$\displaystyle c'=0$}
\item derivacija umno\v{s}ka konstante i funkcije: \fbox{$\displaystyle (c\cdot f)'=c\cdot f'$}
\item derivacija zbroja i razlike: \fbox{$\displaystyle (f\pm g)'=f'\pm g'$}
\item derivacija umno\v{s}ka: \fbox{$\displaystyle (f\cdot g)'=f'\cdot g+f\cdot g'$}
\hspace{5mm}
\fbox{$\displaystyle (f\cdot g\cdot h)'=f'gh+fg'h+fgh'$}
\item derivacija produkta: \fbox{$\displaystyle \left( \frac fg\right)'=\frac{f'\cdot g-f\cdot g'}{g^2} $}
\end{itemize}

\vspace{4mm}
\noindent
\underline{\textbf{Derivacije elemetarnih funkcija:}}\\
\begin{center}
\begin{tabular}{|rcl|rcl|}
\hline
$\displaystyle x'$  & = &  $1$ & $(x^n)'$  & = &  $n\cdot x^{n-1}$\\ 
$\displaystyle \left(\frac 1x\right)'$  & = &  $\displaystyle -\frac 1{x^2}$ & $\displaystyle \left(\frac 1{x^n}\right)'$  & = &  $\displaystyle -\frac n{x^{n+1}}$\\ 
$\displaystyle \left(\sqrt{x}\right)'$  & = &  $\displaystyle \frac1{2\sqrt x}$ & $\displaystyle
\left(\sqrt[n]{x}\right)'$  & = &  $\displaystyle \frac1{n\cdot \sqrt[n]{x^{n-1}}}$\\ 
\hline
$\displaystyle \left(e^x\right)'$  & = &  $\displaystyle e^x$ & $\displaystyle \left(a^x\right)'$  & = &  $\displaystyle a^x\cdot \ln a$\\ 
$\displaystyle (\ln x)'$  & = &  $\displaystyle \frac 1x$ & $\displaystyle (\log_a x)'$  & = &  $\displaystyle \frac 1{x \cdot \ln a}$\\ 
\hline
$\displaystyle (\sin x)'$  & = &  $\displaystyle \cos x$ & $\displaystyle (\cos x)'$  & = &  $\displaystyle -\sin x$\\
$\displaystyle (\textrm{tg}\hspace{1mm}x)'$  & = &  $\displaystyle \frac 1{\cos ^2 x}$ & $\displaystyle (\textrm{cg}\hspace{1mm}x)'$  & = &  $\displaystyle \frac 1{\sin ^2 x}$\\
$\displaystyle (\sin^{-1} x)'$  & = &  $\displaystyle \frac1{\sqrt{1-x^2}}$ & $\displaystyle (\cos^{-1} x)'$  & = &  $\displaystyle -\frac1{\sqrt{1-x^2}}$ \\
$\displaystyle (\textrm{tg}^{-1}\hspace{1mm}x)'$  & = &  $\displaystyle \frac1{1+x^2}$ & $\displaystyle (\textrm{ctg}^{-1}\hspace{1mm}x)'$  & = &  $\displaystyle -\frac1{1+x^2}$\\
\hline
$\displaystyle (\sinh x)'$  & = &  $\displaystyle \cosh x$ & $\displaystyle (\cosh x)'$  & = &  $\displaystyle \sinh x$\\
$\displaystyle (\textrm{tgh}\hspace{1mm}x)'$ & = & $\displaystyle \frac1{\cosh^2x}$ & $\displaystyle (\textrm{ctgh}\hspace{1mm}x)'$ & = & $\displaystyle -\frac1{\sinh^2x}$\\
$\displaystyle (\textrm{sinh}^{-1}\hspace{1mm}x)'$ & = & $\displaystyle \frac1{\sqrt{1+x^2}}$ & $\displaystyle (\textrm{tgh}^{-1}\hspace{1mm}x)'$ & = & $\displaystyle \frac1{1-x^2}$\\
$\displaystyle (\cosh^{-1}\hspace{1mm}x)'$ & = & $\displaystyle \frac1{\sqrt{x^2-1}}$ & $\displaystyle (\textrm{ctgh}^{-1}\hspace{1mm}x)'$ & = & $\displaystyle -\frac1{x^2-1}$\\
\hline
\end{tabular}
\end{center}

\hfill{\fbox{\tiny{v1.0 by \textsf{tkr}}}}
\newpage
\begin{center}
\fbox{\fbox{{\Large \textbf{Integrali}}}}
\end{center}

\noindent
\textbf{\underline{Op\'{c}a pravila integriranja:}}
\begin{itemize}
\item ako je $\displaystyle \int f(x)\textrm dx =F(x)+c$ tada je $F'=f$
\item \fbox{$\displaystyle \int c\cdot f(x) \textrm dx=c\cdot \int f(x)\textrm dx$}
\item\fbox{$\displaystyle \int \textrm dx =x+c$}
\item integral zbroja i razlike: \fbox{$\displaystyle \int [ f(x)\pm g(x)] \textrm dx = \int f(x) \textrm dx \pm \int g(x) \textrm dx$}
\item pravilo supstitucije: ako je $x=\varphi (t)$, tada je: 
	\fbox{$\displaystyle \int f(x)\textrm dx=\int f[\varphi(t)]\cdot \varphi '(t)\textrm dt$}
\item parcijalna integracija: 
	\fbox{$\displaystyle \int (f\cdot g')=f\cdot g-\int (f'\cdot g)$}
\end{itemize}


\vspace{3mm}
\noindent
\underline{\textbf{Tablica osnovnih integrala}}
{\scriptsize(kod svakog integrala s desne strane jednakosti treba dodati konstantu $c$)}
\begin{center}
\begin{tabular}{|ccl|ccl|}
\hline
$\displaystyle \int x^n \textrm dx$ & = & $\displaystyle \frac{x^{n+1}}{n+1}\hspace{2mm} (n\neq 1)$ & 
$\displaystyle \int \frac{\textrm dx }x$ & = & $\displaystyle \ln |x|$\\
\hline

$\displaystyle \int e^x\textrm dx $ & = & $\displaystyle e^x$ & $\displaystyle \int a^x\textrm dx
$ & = & $\displaystyle \frac{a^x}{\ln a}$\\
\hline

$\displaystyle \int \sin x \hspace{1mm}\textrm dx $ & = & $\displaystyle -\cos x$ & $\displaystyle \int \cos x \hspace{1mm}\textrm dx $ & = & $\displaystyle \sin x$\\

$\displaystyle \int \textrm{tg}\hspace{1mm}x\hspace{1mm}\textrm dx $ & = & $\displaystyle -\ln|\cos x|$  &
$\displaystyle \int \textrm{ctg}\hspace{1mm}x\hspace{1mm}\textrm dx $ & = & $\displaystyle \ln|\sin x|$\\

$\displaystyle \int \frac{\textrm dx}{\cos^2 x}$ & = & $\displaystyle \textrm{tg}\hspace{1mm}x $  & $\displaystyle \int
\frac{\textrm dx}{\sin^2 x}$ & = & $\displaystyle -\textrm{ctg}\hspace{1mm}x $\\
\hline

$\displaystyle \int \sinh x\hspace{1mm}\textrm dx$ & = & $\displaystyle  \cosh x$  & $\displaystyle \int \cosh
x\hspace{1mm}\textrm dx$ & = & $\displaystyle  \sinh x$\\

$\displaystyle \int \textrm{tgh}\hspace{1mm}x\textrm dx $ & = & $\displaystyle  \ln |\cosh x|$  & $\displaystyle \int
\textrm{ctgh}\hspace{1mm}x\textrm dx $ & = & $\displaystyle  \ln |\sinh x|$\\

$\displaystyle \int \frac{\textrm dx }{\cosh^2 x}$ & = & $\displaystyle \textrm{tgh}\hspace{1mm}x$  & $\displaystyle
\int \frac{\textrm dx }{\sinh^2 x}$ & = & $\displaystyle -\textrm{ctgh}\hspace{1mm}x$\\
\hline

$\displaystyle \int \frac{\textrm dx }{a^2+x^2}$ & = & $\displaystyle \frac 1a \cdot \textrm{tg}^{-1}\frac xa$  & $\displaystyle \int \frac{\textrm dx }{\sqrt{a^2-x^2}}$ & = & $\displaystyle \sin^{-1}\frac xa$\\

$\displaystyle \int \frac{\textrm dx }{a^2-x^2}$ & = & $\displaystyle \frac 1a \cdot \textrm{tgh}^{-1}\frac xa$ & $\displaystyle \int \frac{\textrm dx }{\sqrt{a^2+x^2}}$ & = & $\displaystyle \sinh^{-1}\frac xa$  \\

$\displaystyle \int \frac{\textrm dx }{x^2-a^2}$ & = & $\displaystyle -\frac 1a \cdot \textrm{ctgh}^{-1}\frac xa$ & $\displaystyle \int \frac{\textrm dx }{\sqrt{x^2-a^2}}$ & = & $\displaystyle \cosh^{-1}\frac xa$\\
\hline

\end{tabular}
\end{center}

\vspace{3mm}
\noindent
\underline{\textbf{Odre\dj{}eni integral:}}
\begin{itemize}
\item ako je $F'=f$ onda: \fbox{$\displaystyle \int_a^bf(x)\textrm dx =F(b)-F(a)$}
\end{itemize}


\end{document}
