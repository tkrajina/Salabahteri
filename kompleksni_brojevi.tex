\documentclass[8pt,oneside,a4paper]{report}
\linespread{2}
\title{kompleksni brojevi}
% \renewcommand{\familydefault}{cmss}
\usepackage[croatian]{babel}
% \usepackage[dvips]{color}
% \usepackage[dvips]{rotating}
\usepackage{epsfig}
\usepackage{floatflt}
\addtolength{\hoffset}{-2cm}
\addtolength{\voffset}{-2.5cm}
\addtolength{\textwidth}{3cm}
\addtolength{\textheight}{4cm}
\pagestyle{empty}

\begin{document}

\begin{center}
\fbox{\fbox{{\Large \textbf{Kompleksni brojevi}}}}
\end{center}

\begin{floatingfigure}[r]{60mm}
\epsfig{figure=fig/gaussova_ravnina.eps}\\
\scriptsize{\textbf{slika:} geometrijski prikaz kompleksnog broja}
\end{floatingfigure}
\noindent
\textbf{Imaginarna jedinica} $i$:\hspace{5mm}\fbox{$\displaystyle i=\sqrt{-1}$}

\noindent
\textbf{Op\'{c}i oblik} kompleksnog broja:\hspace{5mm}
\fbox{$\displaystyle z=a+b\cdot i$}

\noindent
\textbf{Konjugirano kompleksni broj} broju $z=a+bi$:\\
\hspace*{25mm}
\fbox{$\displaystyle \overline z=a-bi$}

\noindent
\textbf{Realni i imaginarni dio} kompleksnog broja:\\ 
\hspace*{25mm}
	\fbox{$\displaystyle \textrm{Re}(a+b\cdot i)=a$}
	\hspace{5mm}
	\fbox{$\displaystyle \textrm{Im}(a+b\cdot i)=b$}

\noindent
\textbf{Jednakost kompleksnih brojeva}:\\
	\hspace*{10mm}
	\fbox{$\displaystyle a+b\cdot i=c+d\cdot i \textrm{ ako je } a=c \textrm{ i } b=d$}

\begin{center}
\fbox{{\large \textbf{Ra\v{c}unske operacije s kompleksnim brojevima}}}
\end{center}

\noindent
\textbf{Zbrajanje i oduzimanje kompleksnih brojeva}:\hspace{5mm}\fbox{$\displaystyle (a+bi)\pm (c+di)=(a\pm c)+(b\pm d)\cdot i$}

% \noindent
% \textbf{Mno\v{z}enje s konstantom}:\\
% \hspace*{25mm}
% \fbox{$\displaystyle \alpha\cdot (a+bi)=\alpha a+\alpha bi$}

\noindent
\textbf{Mno\v{z}enje kompleksnih brojeva}:\hspace{5mm}\fbox{$\displaystyle (a+bi)\cdot (c+di)=(a\cdot c-b\cdot d)+(a\cdot d+b\cdot c)\cdot i$}

\noindent
\textbf{Dijeljenje kompleksnih brojeva}:\hspace{5mm}\fbox{$\displaystyle \frac{a+bi}{c+di}=\frac{ac+bd}{c^2+d^2}+\frac{cb-ad}{c^2+d^2}\cdot i$}

\noindent
\textbf{Potencije imaginarne jedinice}:\hspace{5mm}\fbox{$i^{4n}=1$}
\hspace{5mm}
\fbox{$i^{4n+1}=i$}
\hspace{5mm}
\fbox{$i^{4n+2}=-1$}
\hspace{5mm}
\fbox{$i^{4n+3}=-i$}

\begin{center}
\fbox{{\large \textbf{Trigonometrijski oblik kompleksnog broja}}}
\end{center}
\begin{floatingfigure}[r]{55mm}
\epsfig{figure=fig/trig_kompl.eps}\\
\end{floatingfigure}
\noindent
\textbf{Trigonometrijski oblik} broja $z=a+bi$:\\
\hspace*{20mm}\fbox{$\displaystyle z=r\cdot (\cos \varphi +i\cdot \sin \varphi)$}\\
\hspace*{10mm}$r$\dots modul broja $z$:\hspace{5mm}\fbox{$\displaystyle r=\sqrt{a^2+b^2}$}\\
\noindent
\hspace*{10mm}$\varphi$ \dots argument broja $z$:\hspace{5mm}\fbox{$\textrm{tg}\hspace{1mm}\varphi=\frac ba$}

\noindent
{\scriptsize Neka je $z_k=r_k\cdot (\cos \varphi_k+i\cdot \sin\varphi_k)$ za $k=1,2$}

\noindent
\textbf{Mno\v{z}enje kompleksnih brojeva}:\hspace{5mm}\fbox{$\displaystyle z_1\cdot z_2=r_1r_2\cdot [\cos(\varphi_1+\varphi_2)+i\cdot \sin(\varphi_1+\varphi_2)]$}

\noindent
\textbf{Dijeljenje kompleksnih brojeva}:\hspace{5mm}\fbox{$\displaystyle \frac{z_1}{z_2}=\frac{r_1}{r_2}\cdot [\cos(\varphi_1-\varphi_2)+i\cdot \sin(\varphi_1-\varphi_2)]$}

\noindent
\textbf{Potenciranje kompleksnog broja} $z=r\cdot (\cos \varphi+i\cdot \cdot \sin\varphi)$\\
\hspace*{25mm}\fbox{$\displaystyle z^n=r^n\cdot  [\cos n\varphi+i\cdot \sin n\varphi]$}

\noindent
\textbf{Korjenovanje kompleksnog broja} $z=r\cdot (\cos \varphi+i\cdot \sin\varphi)$\\
\hspace*{25mm}\fbox{$\displaystyle \sqrt[n]{z}=\sqrt[n]{r}\cdot
\left( \cos\frac{\varphi+2k\pi}{n}+i\cdot \sin\frac{\varphi+2k\pi}{n}\right)$\hspace{6mm}$k=0,1,2\dots n-1$}

\hfill{\fbox{\tiny{v1.01 by \textsf{tkr}}}}
\end{document}
