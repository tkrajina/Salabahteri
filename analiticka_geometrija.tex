\documentclass[9pt,oneside,a4paper]{report}
% \linespread{1.3}
\title{qq}
\renewcommand{\familydefault}{cmss}
\usepackage[croatian]{babel}
% \usepackage[dvips]{color}
% \usepackage[dvips]{rotating}
\usepackage{epsfig}
\usepackage{floatflt}
\addtolength{\hoffset}{-2cm}
\addtolength{\voffset}{-2.5cm}
\addtolength{\textwidth}{3cm}
\addtolength{\textheight}{4cm}
\pagestyle{empty}

\begin{document}

\begin{center}
\fbox{\fbox{{\huge \textsf{Analiti\v{c}ka geometrija ravnine -- osnovne formule}}}}
\end{center}

\begin{floatingfigure}[r]{7cm}
\epsfig{figure=fig/ksustav.eps}
\end{floatingfigure}

\vspace{3mm}
\noindent \textbf{Udaljenost izme\dj{}u dvije to\v{c}ke} $T_1(x_1,y_1)$ i $T_2(x_2,y_2)$:\\
\hspace*{10mm}\fbox{$\displaystyle d(T_1,T_2)=\sqrt{(x_2-x_1)^2+(y_2-y_1)^2}$}

\vspace{3mm}
\noindent \textbf{Dijeljenje du\v{z}ine} $\overline{T_1T_2}$ to\v{c}kom
$T(x_t,y_t)$ u omjeru $-\lambda =\frac{|T_1T|}{|TT_2|}$:
\hspace{5mm}\fbox{$x_t=\frac{x_1-\lambda x_2}{1-\lambda}$}
\hspace{5mm}\fbox{$y_t=\frac{y_1-\lambda y_2}{1-\lambda}$}

\vspace{9mm}
\noindent
\fbox{\textbf{\Large{Pravac u koordinatnom sustavu}}}

\vspace{3mm}
\begin{floatingfigure}[r]{5cm}
\vspace*{1cm}
\epsfig{figure=fig/pravac.eps}
\end{floatingfigure}

\noindent \textbf{Eksplicitna jednad\v{z}ba} pravca:\hspace{5mm} 
% \hspace*{30mm}
\fbox{$y=k\cdot x+l$}\\
% \hspace*{5mm}$k$ \dots koeficijent smjera\\
% \hspace*{5mm}$l$ \dots odsje\v{c}ak na osi $y$
\\*
\hspace*{5mm}$\displaystyle \textrm{\textsf{pravac je}}\left\{
\begin{array}{ll}
\textrm{\textsf{rastu\'{c}i}} & \textrm{\textsf{ako je $k>0$}}\\
\textrm{\textsf{padaju\'{c}i}} & \textrm{\textsf{ako je $k<0$}}\\
\textrm{\textsf{vodoravan}} & \textrm{\textsf{ako je $k=0$}}\\
\end{array} \right.$

\vspace{3mm}
\noindent \textbf{Implicitna jednad\v{z}ba} pravca:\hspace{5mm}
\fbox{$A\cdot x+B\cdot y+C=0$}

\vspace{3mm}
\noindent
\textbf{Segmentna jednad\v{z}ba pravca};
ako pravac na osima $O_x$ i $O_y$ sije\v{c}e odsje\v{c}ke $m$ i $n$: 
\hspace*{5mm}\fbox{$\displaystyle \frac xm +\frac yn =1$}

\vspace{3mm}
\noindent
\textbf{Pravac zadan s to\v{c}kom $T(x_1,y_1)$ i koeficijentom smjera $k$} 
\hspace*{30mm} \fbox{$y-y_1=k\cdot (x-x_1)$}

\vspace{3mm}
\noindent
\textbf{Pravac zadan s dvije to\v{c}ke} 
$T_1(x_1,y_1)$ i $T_2(x_2,y_2)$: \hspace{5mm}
\fbox{$\displaystyle y-y_1=\frac{y_2-y_1}{x_2-x_1}(x-x_1)$}
\hspace{5mm}\fbox{$k=\frac{y_2-y_1}{x_2-x_1}$}

\vspace{3mm}
\noindent
\textbf{Udaljenost $\delta$ to\v{c}ke do pravca}; to\v{c}ka $T(x_0,y_0)$ i
pravac $Ax+By+C=0$:
\hspace*{4mm}\fbox{$\displaystyle \delta=\Bigg|\frac{A\cdot x_0+B\cdot y_0+C}{\sqrt{A^2+B^2}}\Bigg|$\hspace{1mm}}

\vspace{4mm}
\noindent \textbf{Kut $\varphi$ izme\dj{}u pravca $y=kx+l$ i osi $O_x$}: \hspace*{5mm}\fbox{$\textrm{tg}\varphi=k$}

\vspace{3mm}
\noindent \textbf{Kut $\varphi$ izme\dj{}u dva pravca} s koeficijentima
smjera $k_1$ i $k_2$:\\
\hspace*{30mm}\fbox{$\displaystyle \textrm{tg}\varphi =\frac{k_2-k_1}{1+k_1k_2}$}\hspace{5mm} 
{\scriptsize (ako dobijemo $\varphi>90^\circ$ tada kao rezultat uzimamo
$180^\circ-\varphi$)}

\begin{floatingfigure}[r]{6cm}
\epsfig{figure=fig/kruznica.eps}
\end{floatingfigure}

\vspace{3mm}
\noindent
\textbf{Uvjet paralelnosti i okomitosti} dvaju pravaca s koeficijentima
smjera $k_1$ i $k_2$:\\
\hspace*{5mm}$\displaystyle \textrm{\textsf{pravci $p_1$ i $p_2$ su}}\left\{
\begin{array}{ll}
	\textrm{\textsf{paralelni $p_1||p_2$,}} & \textrm{\textsf{ako je \fbox{$k_1=k_2$}}}\\
	\textrm{\textsf{okomiti $p_1\bot p_2$,}} & \textrm{\textsf{ako je
	\fbox{$k_1=-\frac 1{k_2}$}}}
\end{array} \right.$
\vspace{3mm}

\vspace{3mm}
\noindent
\fbox{\textbf{\Large{Kru\v{z}nica}}}

\vspace{3mm}
\noindent
\textbf{Jednad\v{z}ba kru\v{z}nice} sa sredi\v{s}tem $S(p,q)$ i polumjerom $r$:\\ \hspace*{25mm}\fbox{$\displaystyle (x-p)^2+(y-q)^2=r^2$}

\vspace{3mm}
\noindent \textbf{Jednad\v{z}ba centralne kru\v{z}nice}:\hspace{5mm}\fbox{$\displaystyle x^2+y^2=r^2$}

\noindent
\textbf{Jednad\v{z}ba tangente} na kru\v{z}nicu s dirali\v{s}tem
$D(x_0,y_0)$: \hspace*{5mm}\fbox{$(x-p)(x_0-p)+(y-q)(y_0-q)=r^2$}

\vspace{1mm}
\noindent
\textbf{Uvjet da bi pravac $y=kx+l$ bio tangenta na kru\v{z}nicu:}
\hspace{5mm}\fbox{$r^2\cdot (k^2+1)=(k\cdot p-q+l)^2$}

\vspace{5mm}
\begin{floatingfigure}[r]{7cm}
\epsfig{figure=fig/elipsa.eps}
\end{floatingfigure}

\vspace{3mm}
\noindent
\fbox{\textbf{\Large{Elipsa}}}

\vspace{3mm}\noindent
\textbf{Svojstvo proizvoljne to\v{c}ke $T$ na elipsi:}\\
\hspace*{20mm}\fbox{$d(T,F_1)+d(T,F_2)=2\cdot a=const$}


\vspace{3mm}\noindent
\textbf{Jednad\v{z}ba elipse} ($a>b$):\\
\hspace*{10mm}\fbox{$\displaystyle \frac{x^2}{a^2}+\frac{y^2}{b^2}=1$}
\hspace*{5mm}\fbox{$\displaystyle b^2\cdot x^2+a^2\cdot y^2=a^2\cdot b^2$}

\vspace{3mm}\noindent
\noindent
\textbf{Jednad\v{z}ba tangente} na elipsu s dirali\v{s}tem
$D(x_0,y_0)$:\\
\hspace*{20mm}\fbox{$\displaystyle \frac{x\cdot x_0}{a^2}+\frac{y\cdot y_0}{b^2}=1$}

\vspace{1mm}
\noindent
\textbf{Uvjet da bi pravac $y=kx+l$ bio tangenta} na elipsu:
\hspace{5mm}\fbox{$a^2k^2+b^2=l^2$}

\vspace{1mm}
\noindent
\textbf{Linearni ekscentricitet:}\hspace{5mm}\fbox{$e=\sqrt{a^2-b^2}$}

\vspace{1mm}
\noindent
\textbf{Numeri\v{c}ki ekscentricitet:}
\hspace{5mm}\fbox{$\varepsilon=\frac ea$}

\begin{floatingfigure}[r]{6cm}
\epsfig{figure=fig/hiperbola.eps}
\end{floatingfigure}

\vspace{3mm}
\noindent
\fbox{\textbf{\Large{Hiperbola}}}

\vspace{3mm}\noindent
\textbf{Svojstvo proizvoljne to\v{c}ke $T$ na hiperboli:}\\
\hspace*{20mm}\fbox{$|d(T,F_1)-d(T,F_2)|=2\cdot a=const$}


\vspace{3mm}\noindent
\textbf{Jednad\v{z}ba hiperbole:}\\
\hspace*{10mm}\fbox{$\displaystyle \frac{x^2}{a^2}-\frac{y^2}{b^2}=1$}
\hspace*{5mm}\fbox{$\displaystyle b^2\cdot x^2-a^2\cdot y^2=a^2\cdot b^2$}

\vspace{3mm}\noindent
\noindent
\textbf{Jednad\v{z}ba tangente} na hiperbolu s dirali\v{s}tem
$D(x_0,y_0)$:\\
\hspace*{20mm}\fbox{$\displaystyle \frac{x\cdot x_0}{a^2}-\frac{y\cdot y_0}{b^2}=1$}

\vspace{3mm}\noindent
\noindent
\textbf{Asimptote na hiperbolu:}\\
\hspace*{10mm}\fbox{$\displaystyle p_1\dots y=-\frac ba \cdot x$}
\hspace*{5mm}\fbox{$\displaystyle p_2\dots y=\frac ba \cdot x$}

\vspace{1mm}
\noindent
\textbf{Uvjet da bi pravac $y=kx+l$ bio tangenta} na hiperbolu:
\hspace{5mm}\fbox{$a^2k^2-b^2=l^2$}

\vspace{1mm}
\noindent
\textbf{Linearni ekscentricitet:}\hspace{5mm}\fbox{$e=\sqrt{a^2+b^2}$}

\vspace{1mm}
\noindent
\textbf{Numeri\v{c}ki ekscentricitet:}
\hspace{5mm}\fbox{$\varepsilon=\frac ea$}

\begin{floatingfigure}[r]{6cm}
\epsfig{figure=fig/parabola.eps}
\end{floatingfigure}

\vspace{3mm}
\noindent
\fbox{\textbf{\Large{Parabola}}}

\vspace{3mm}\noindent
\textbf{Svojstvo proizvoljne to\v{c}ke $T$ na paraboli:}\\
\hspace*{20mm}\fbox{$d(T,F)=d(T,r)$}


\vspace{3mm}\noindent
\textbf{Jednad\v{z}ba parabole:}\\
\hspace*{20mm}\fbox{$\displaystyle y^2=2p\cdot x$}

\vspace{3mm}\noindent
\noindent
\textbf{Jednad\v{z}ba tangente} na parabolu s dirali\v{s}tem
$D(x_0,y_0)$:\\
\hspace*{20mm}\fbox{$y\cdot y_0=p\cdot (x+x_0)$}

\vspace{3mm}\noindent
\noindent
\textbf{Jednad\v{z}ba ravnalice} parabole:\hspace{3mm}\fbox{$\displaystyle r\dots x=-\frac p2 $}

\vspace{1mm}
\noindent
\textbf{Uvjet da bi pravac $y=kx+l$ bio tangenta} na parabolu:
\hspace{5mm}\fbox{$p=2kl$}

\hfill{\fbox{\tiny{v1.2 by \textsf{tkr}}}}

\end{document}
